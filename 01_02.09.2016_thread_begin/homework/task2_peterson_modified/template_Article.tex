\documentclass[]{article}

\usepackage[utf8]{inputenc}
\usepackage[T1,T2A]{fontenc}
\usepackage[english,russian]{babel}
\usepackage{amsmath, amssymb}
\usepackage{enumerate}
\usepackage{enumitem}
\usepackage{amsthm}
\usepackage{multirow}
\usepackage{centernot}
\usepackage{listings}

%opening
\title{Домашнее задание на 09 сентября. Задача 2.}
\author{Ульянин Дмитрий}
\date{09.09.2016}


\begin{document}

\maketitle

Пусть, напимер, поток $0$ совершил запись \texttt{victim.store(0)}, а потом был прерван потоком $1$. Поток $1$ совершил запись \texttt{victim.store(1)}, и вошел в критическую секцию, так как не выполняется want[0].

Когда управление будет возвращено потоку $0$, он тоже успешно войдет в критическую секцию, так как \texttt{victim} уже перезаписана и не выполняется \texttt{victim == 0}

Ответ: \texttt{
	\newline 0: victim = 0;
	\newline 1: victim = 1;
	\newline 1: want[1] = 1;
	\newline 1: load(victim) == 1;
	\newline 1: load(want[0]) == 0;
	\newline 1: вход в критическую секцию;
	\newline 0: want[0] = 0
	\newline 0: load(victim) == 1;
	\newline 0: load(want[1]) == 1;
	\newline 0: вход в критическую секцию}



\end{document}
