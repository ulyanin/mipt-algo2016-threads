\documentclass[]{article}

\usepackage[utf8]{inputenc}
\usepackage[T1,T2A]{fontenc}
\usepackage[english,russian]{babel}
\usepackage{amsmath, amssymb}
\usepackage{enumerate}
\usepackage{enumitem}
\usepackage{amsthm}
\usepackage{multirow}
\usepackage{centernot}

%% for case with sqare bracket
\makeatletter
\newenvironment{sqcases}{%
	\matrix@check\sqcases\env@sqcases
}{%
\endarray\right.%
}
\def\env@sqcases{%
	\let\@ifnextchar\new@ifnextchar
	\left\lbrack
	\def\arraystretch{1.2}%
	\array{@{}l@{\quad}l@{}}%
}
\makeatother


%opening
\title{Домашнее задание на 16 сентября, задача 3}
\author{Ульянин Дмитрий}
\date{14.09.2016}

\newcommand{\olen}{\overline{len}}
\newcommand{\onum}{\overline{num}}
\newcommand{\omaxlen}{\overline{maxlen}}
\newcommand{\osumnum}{\overline{sumnum}}
\newcommand{\logn}{\log n}
\newcommand{\Ologn}{O(\logn)}
\newcommand{\reven}[1]{#1^{2*}}
\newcommand{\whereRu}{\texttt{Где }}
\newcommand{\ifRu}{\texttt{Если}}
\newcommand{\divs}{\vdots}
\newcommand{\undivs}{\centernot\vdots}
\relpenalty=10000 
\binoppenalty=10000


\begin{document}

\maketitle

Пусть $k = \texttt{std::thread::hardware\_concurrency}$.

Тогда утверждается, что при $n > k$ \texttt{ticket\_spinlock} будет работать неоптимально.

Действительно, пусть какой-то потом получил ticket и зашел в короткую критическую секцию. По условию поток может исполнить ее много раз за один квант времени, но вместо того, чтобы продолжить исполняться, ему выдается новый билет с номером, большим чем у других потоков. (Происходит это при ${n > k}$, т.к. иначе поток сразу сможет исполняться на свободном потоке ядра процессора).

При этом этот поток теперь будет ждать, пока не придет его очередь снова. Мы теряем время из-за того, что планировщик работает не очень быстро, ибо на это требуется несколько прерываний на процессоре и какие-то вычисления. А так же теряем время на то, что на каждый заход в критическую секцию каким-то процессом \texttt{ticket\_spinlock} будет выполнять не очень тривиальные операции.

\texttt{test\_and\_set\_spinlock} не подвержен этой проблеме, т.к. если процесс выполнил критическую секцию, но квант времени еще не израсходовал, то он просто войдет в \texttt{lock()} и  \texttt{locked\_.exchange(true)} вернет \texttt{false} и поток продолжит исполнять критические секции, пока квант не истечет.


\end{document}
